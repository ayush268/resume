%-------------------------------------------------------------------------------
%	SECTION TITLE
%-------------------------------------------------------------------------------
\cvsection{Projects}


%-------------------------------------------------------------------------------
%	CONTENT
%-------------------------------------------------------------------------------
\begin{cventries}

%---------------------------------------------------------
  \cventry
    {Under Prof. Satyadev Nandakumar} % Role
    {\href{https://github.com/ayush268/InHs}{InHs}: An interpreter in Haskell} % Event
    {IIT Kanpur} % Location
    {Sept 2019 - Nov 2019} % Date(s)
    {
      \begin{cvitems} % Description(s)
      \item {Developed an interpreter from scratch in \textit{Haskell} interpreting a simple kernel language such as \textit{Oz}.}
      \item {InHs supported all the basic features of a declarative sequential and concurrent language such as application of non-suspendable and suspendable statements, unification of variables and values, maintenance of a single assignment store and a semantic multi-stack.}
      \item {InHs also extends the functionality of basic kernel languages to include features such as Lazy and concurrent executions via trigger store and Non-determinism using Wait Statement, IsDetermined Statement and Message passing model via mutable store.}
      \end{cvitems}
    }%\vspace*{-\baselineskip}

%---------------------------------------------------------

%---------------------------------------------------------
  %\cventry
  %  {Under Prof. Brendan Dolan-Gavitt (NYU) and Prof. Subhajit Roy} % Role
  %  {Integrating bug synthesis into a bug finding competition} % Event
  %  {IIT Kanpur} % Location
  %  {June 2018 - Dec 2018} % Date(s)
  %  {
  %    \begin{cvitems} % Description(s)
  %      \item {Worked on a Bug Injection Tool ``Apocalypse" used to inject bugs in C code which are difficult to locate by automated security verification tools.}
  %      \item {Implemented techniques for crashing the buggy C programs from Apocalypse and disguise the bugs rendering them to locate.}
% %       \item {Implemented techniques to disguise the bugs to make it difficult for users to locate them.}
  %      \item {Injected bugs in ``grep" code based on the implemented techniques and found 50\% success rate in bug finding competition \href{https://rode0day.mit.edu/results/4}{``Rode0day"}.}
  %    \end{cvitems}
  %  }

%---------------------------------------------------------


%---------------------------------------------------------
  \cventry
    {Under Prof. Sandeep Shukla} % Role
    {Certification: Issuing tamper-proof certificates} % Event
    {IIT Kanpur} % Location
    {Apr 2019} % Date(s)
    {
      \begin{cvitems} % Description(s)
        \item {A Blockchain based service issuing certificates which once issued cannot be altered by anyone, accomplished by making the hash of the certificate as part of the public \textit{Ethereum} blockchain rendering it easily accessible and verifiable by anyone in the world.}
        \item {The service works on a multi-level hierarchical system in which the instructor/manager can issue certificates for the students/employees and once issued, no one can alter them and anyone in the world can verify the certificates, thus having a wide array of applications.}
      \end{cvitems}
    }

%---------------------------------------------------------


%---------------------------------------------------------
  \cventry
    {Under Prof. Amey Karkare} % Role
    {PyGo: A compiler for Golang in Python} % Event
    {IIT Kanpur} % Location
    {Feb 2019 - Apr 2019} % Date(s)
    {
      \begin{cvitems} % Description(s)
      \item {Developed a \textit{compiler from scratch in Python} with \textit{source language as Golang} and \textit{target as x86\_32}.}
      \item {PyGo supported all the basic features of Golang along with the advanced features such as optimizations, efficient register allocation, short circuiting, type checking and inference.}
      \end{cvitems}
    }

%---------------------------------------------------------


%---------------------------------------------------------
  \cventry
    {Undergraduate Project, Prof. Debadatta Mishra} % Role
    {Exploiting Binaries using Return Oriented Programming} % Event
    {IIT Kanpur} % Location
    {Jan 2019 - Apr 2019} % Date(s)
    {
      \begin{cvitems} % Description(s)
        \item {Worked with x86\_64 binaries to invoke any particular syscall by performing static analysis and using concepts of Return Oriented Programming.}
        \item {Developed a tool for building a ROP chain from ROP gadgets to exploit the 64-bit binaries by invoking a particular syscall.}
      \end{cvitems}
    }

%---------------------------------------------------------


%---------------------------------------------------------
  \cventry
    {Research Project, Prof. Subhajit Roy} % Role
    {InVaSion: Input Validation with Symbolic Execution} % Event
    {IIT Kanpur} % Location
    {May 2018 - Nov 2018} % Date(s)
    {
      \begin{cvitems} % Description(s)
      \item {Worked on forking-based symbolic execution engines to tackle the problem of Path Explosion with goal of increasing code and branch coverage for programs requiring formatted inputs such image readers or video players.}
      \item {Formulated a framework to represent the formatted input as a FSA, an input specification language (ISL) to be used by a software.}
        \item {Developed a tool in Ruby to parse and compile the ISL, and appending this ISL to the input program, generating a set of logical constraints on execution by symbolic execution engine, with the goal of guiding symbolic execution tool to produce better inputs.}
        \item {Improved branch coverage by greater than 150\% on the programs in Siemens Test Suite, MiBench and Tiff library which require formatted input.}
      \end{cvitems}
    }%\vspace{-4mm}

%---------------------------------------------------------


%---------------------------------------------------------
%  \cventry
%    {Programming Club} % Role
%    {Ethical Hacking: Learnt Basic Exploits for participating in CTFs} % Event
%    {IIT Kanpur} % Location
%    {Summer'2017} % Date(s)
%    {
%      \begin{cvitems} % Description(s)
%        \item {Learnt exploitation techniques like buffer-overflow, integer-overflow, format strings, heap exploits and basics of Cryptography.}
%        \item {Analyzed and exploited x86 binaries using debuggers and knowledge of x86 Assembly language, and performed diverse web attacks like sql-injection, XSS and CSRF.}
%      \end{cvitems}
%    }

%---------------------------------------------------------


%---------------------------------------------------------
  \cventry
    {Association of Computing Activities} % Role
    {Linux From Scratch: Compiled Linux System from Source and Built a Package Manager} % Event
    {IIT Kanpur} % Location
    {Feb 2017 - Apr 2017} % Date(s)
    {
      \begin{cvitems} % Description(s)
        \item {Created a \textit{Customized Linux based operating system} by compiling all packages from scratch using the host machine's kernel.}
        \item {Built a \textit{Package Manager} in python which fetched the dependencies of packages and created a dependency list based on packages installed.} 
      \end{cvitems}
    }

%---------------------------------------------------------
\end{cventries}
