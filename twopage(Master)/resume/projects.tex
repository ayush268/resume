%-------------------------------------------------------------------------------
%	SECTION TITLE
%-------------------------------------------------------------------------------
\cvsection{Projects}


%-------------------------------------------------------------------------------
%	CONTENT
%-------------------------------------------------------------------------------
\begin{cventries}

%---------------------------------------------------------
%  \cventry
%    {NimbleEdge Open Source} % Role
%    {EnvisEdge: Envision Edge on the Cloud} % Event
%    {} % Location
%    {July 2021 - Apr 2022} % Date(s)
%    {
%      \begin{cvitems} % Description(s)
%      \item {EnvisEdge allows users to simulate an edge computing environment to test their ideas and models before putting them in place on the edge.}
%      \item {A user can setup an environment of their choice with any arbitrary hardware constraints such as RAM, CPU and more.}
%      \item {Simulate devices on the scale of 100s of millions, testing algorithms \& models on a large scale, something which only Big Techs could do.}
%      \item {Focussed technologies and areas: Scala (with Akka), Golang, Python, Distributed Edge Computing, Federated Learning.}
%      \end{cvitems}
%    }

%---------------------------------------------------------


%---------------------------------------------------------
%  \cventry
%    {Under Prof. Satyadev Nandakumar and Prof. Pramod Subramanyan} % Role
%    {Typed PRNGs: Infuse Type Theory with Random Number Generation} % Event
%    {IIT Kanpur} % Location
%    {Jan 2020 - May 2020} % Date(s)
%    {
%      \begin{cvitems} % Description(s)
%        % TODO
%      \end{cvitems}
%    }

%---------------------------------------------------------


%---------------------------------------------------------
%  \cventry
%    {Under Prof. Manindra Agrawal} % Role
%    {\href{https://github.com/ayush268/Crypto_Game}{Modern Cryptology}: understanding, breaking and making Cryptographic Ciphers} % Event
%    {IIT Kanpur} % Location
%    {Jan 2020 - Apr 2020} % Date(s)
%    {
%      \begin{cvitems} % Description(s)
%      \item {Developed multiple techniques for identifying \& breaking commonly used ciphers like Substitution, Vigenere, Permutation-Substitution, etc.}
%      \item {Developed technique and program for \textbf{breaking multi-round DES Cipher and AES Cipher}.}
%      \item {Developed the program for identifying \textbf{RSA Encryption with small exponent} and breaking it using \textbf{Coppersmith’s Attack}.}
%      \item {Dove into depth of KECCAK hash function to identify vulnerability and breaking a weaker version of it.}
%      \end{cvitems}
%    }

%---------------------------------------------------------

%---------------------------------------------------------
  \cventry
    {Under Prof. Satyadev Nandakumar} % Role
    {\href{https://github.com/ayush268/InHs}{InHs}: An interpreter in Haskell} % Event
    {IIT Kanpur} % Location
    {Sept 2019 - Nov 2019} % Date(s)
    {
      \begin{cvitems} % Description(s)
      \item {Developed an \textbf{interpreter from scratch in Haskell} interpreting a simple kernel language such as \textbf{Oz}.}
      \item {InHs supported all the basic features of a declarative sequential and concurrent language such as application of non-suspendable and suspendable statements, unification of variables and values, maintenance of a single assignment store and a semantic multi-stack.}
      \item {InHs also extends the functionality of basic kernel languages to include features such as Lazy and concurrent executions via trigger store and Non-determinism using Wait Statement, IsDetermined Statement and Message passing model via mutable store.}
      \end{cvitems}
    }%\vspace*{-\baselineskip}

%---------------------------------------------------------


%---------------------------------------------------------
%  \cventry
%    {Under Prof. Sandeep Shukla} % Role
%    {\href{https://github.com/ayush268/certification}{Certification}: Issuing tamper-proof certificates} % Event
%    {IIT Kanpur} % Location
%    {Apr 2019} % Date(s)
%    {
%      \begin{cvitems} % Description(s)
%        \item {A \textbf{Blockchain based service issuing certificates} which once issued cannot be altered by anyone, accomplished by making the hash of the certificate as part of the public \textbf{Ethereum} blockchain rendering it easily accessible and verifiable by anyone in the world.}
%        \item {The service \textbf{works on a multi-level hierarchical system} in which the instructor/manager can issue certificates for the students/employees and once issued, no one can alter them and anyone in the world can verify the certificates, thus having a wide array of applications.}
%        \item {Focussed technologies and areas: Solidity, Ruby on Rails, Ethereum Blockchain, Security.}
%      \end{cvitems}
%    }

%---------------------------------------------------------


%---------------------------------------------------------
%  \cventry
%    {Under Prof. Amey Karkare} % Role
%    {\href{https://github.com/hritwik567/PyGo}{PyGo}: A compiler for Golang in Python} % Event
%    {IIT Kanpur} % Location
%    {Feb 2019 - Apr 2019} % Date(s)
%    {
%      \begin{cvitems} % Description(s)
%      \item {Developed a \textbf{compiler from scratch in Python} with \textbf{source language as Golang} and \textbf{target as x86\_32}.}
%      \item {PyGo supported all the basic features of Golang along with the advanced features such as optimizations, efficient register allocation, short circuiting, type checking and inference.}
%      \end{cvitems}
%    }

%---------------------------------------------------------


%---------------------------------------------------------
%  \cventry
%    {Undergraduate Project, Prof. Debadatta Mishra} % Role
%    {\href{https://arxiv.org/abs/2111.03537}{pROP}: Programmable Return Oriented Programming} % Event
%    {IIT Kanpur} % Location
%    {Jan 2019 - Apr 2019} % Date(s)
%    {
%      \begin{cvitems} % Description(s)
%      \item {Worked with \textbf{x86\_64 binaries to invoke any particular syscall} by performing static analysis \& using concepts of Return Oriented Programming.}
%      \item {Developed a \textbf{tool (pROP) for building a ROP chain from ROP gadgets} to exploit the 64-bit binaries by invoking a particular syscall.}
%      \item {Focussed technologies and areas: C/C++, Python, Systems Security - Overflows \& Exploits, Computer Architecture, Operating Systems.}
%      \end{cvitems}
%    }

%---------------------------------------------------------


%---------------------------------------------------------
%  \cventry
%    {Under Prof. Debadatta Mishra} % Role
%    {\href{https://dl.acm.org/doi/10.1145/3338698.3338887}{GemOS}: Building a simple OS (in C) for 64-bit systems} % Event
%    {IIT Kanpur} % Location
%    {July 2018 - Oct 2018} % Date(s)
%    {
%      \begin{cvitems} % Description(s)
%      \item {Developed the \textbf{paging structures} for an execution context as per 64-bit virtual address layout, completing the \textbf{virtual memory management}.}
%      \item {Implemented the essential \textbf{system calls} like write, expand, sleep, clone, etc. and \textbf{exception handlers} like page fault, etc.}
%      \item {Implemented and enabled the support for \textbf{lazy memory allocation} along with functionalities like \textbf{copy-on-write for memory pages}.}
%      \item {Developed \textbf{Signal Handler} and implemented the common signals like SIGSEGV, SIGFPE, SIGALRM, along with a system timer (for alarms).}
%      \item {Developed a \textbf{Round-Robin Scheduler} along with the functionality of setup and cleanup of stack while switching processes.}
%      \end{cvitems}
%    }

%---------------------------------------------------------


%---------------------------------------------------------
  \cventry
    {Under Prof. Brendan Dolan-Gavitt (NYU) and Prof. Subhajit Roy} % Role
    {\href{https://rode0day.mit.edu/}{Rode0day}: Integrating bug synthesis into a bug finding competition} % Event
    {IIT Kanpur} % Location
    {June 2018 - Dec 2018} % Date(s)
    {
      \begin{cvitems} % Description(s)
        \item {Worked on a \textbf{Bug Injection Tool \href{https://dl.acm.org/doi/abs/10.1145/3236024.3236084}{Apocalypse}} used to inject bugs in C code which are difficult to locate by automated security verification tools.}
        \item {Implemented techniques for crashing the buggy C programs from Apocalypse and disguise the bugs rendering them hard to locate.}
        \item {Injected \textbf{bugs in grep code} based on the implemented techniques and \textbf{found 50\% success rate} \href{https://rode0day.mit.edu/results/4}{in bug finding competition Rode0day}.}
        \item {Focussed technologies and areas: C/C++, Security, Formal Methods - Software Verification, Symbolic Execution and Fuzzing.}
      \end{cvitems}
    }

%---------------------------------------------------------




%---------------------------------------------------------
  \cventry
    {Research Project, Prof. Subhajit Roy} % Role
    {\href{https://arxiv.org/abs/2104.01438}{InVaSion}: Input Validation with Symbolic Execution} % Event
    {IIT Kanpur} % Location
    {May 2018 - Nov 2018} % Date(s)
    {
      \begin{cvitems} % Description(s)
      \item {Worked on \textbf{forking-based symbolic execution engines to tackle the problem of Path Explosion} with goal of increasing code and branch coverage for programs requiring formatted inputs such image readers or video players.}
      \item {Formulated a framework to represent the formatted input as a FSA, an input specification language (ISL) to be used by a software.}
        \item {Developed a \textbf{tool in Ruby to parse and compile the ISL, and appending this ISL to the input program}, generating a set of logical constraints on execution by symbolic execution engine, with the goal of guiding symbolic execution tool to produce better inputs.}
        \item {\textbf{Improved branch coverage by greater than 150\%} on programs in Siemens Test Suite, MiBench and Tiff library which require formatted input.}
        \item {Focussed technologies and areas: C/C++, Ruby, Compilers, Formal Methods - Software Verification, Symbolic Execution and Fuzzing.}
      \end{cvitems}
    }%\vspace{-4mm}

%---------------------------------------------------------


%---------------------------------------------------------
%  \cventry
%    {Programming Club} % Role
%    {\href{https://github.com/pclubiitk/crypto-18}{Crypto}: Capture the Flag Contest} % Event
%    {IIT Kanpur} % Location
%    {Oct 2018} % Date(s)
%    {
%      \begin{cvitems} % Description(s)
%      \item {\textbf{Organised and prepared problems} for the Capture the Flag Contest as part of the annual \textbf{Crypto Competition} for the entire campus.}
%      \item {Topics Covered: Overflows, Formal Methods, Template Injections, SQL Injections, Networks, Cryptography.}
%      \end{cvitems}
%    }

%---------------------------------------------------------



%---------------------------------------------------------
%  \cventry
%    {Programming Club} % Role
%    {\href{https://github.com/ayush268/Ethical\_Hacking}{Ethical Hacking}: Learnt Basic Exploits for CTFs} % Event
%    {IIT Kanpur} % Location
%    {Summer'2017} % Date(s)
%    {
%      \begin{cvitems} % Description(s)
%        \item {Learnt exploitation techniques like buffer-overflow, integer-overflow, format strings, heap exploits and basics of Cryptography.}
%        \item {Analyzed and exploited x86 binaries using debuggers and knowledge of x86 Assembly language, and performed diverse web attacks like sql-injection, XSS and CSRF.}
%      \end{cvitems}
%    }

%---------------------------------------------------------


%---------------------------------------------------------
  \cventry
    {Association of Computing Activities} % Role
    {\href{https://ayushb.org/posts/linux-from-scratch/}{Linux From Scratch}: Compiled Linux System from Source and Built a Package Manager} % Event
    {IIT Kanpur} % Location
    {Feb 2017 - Apr 2017} % Date(s)
    {
      \begin{cvitems} % Description(s)
        \item {Created a \textbf{Customized Linux based operating system} by compiling all packages from scratch using the host machine's kernel.}
        \item {Built a \textbf{Package Manager in python} which fetched the dependencies of packages and created a dependency list based on packages installed.} 
      \end{cvitems}
    }

%---------------------------------------------------------
\end{cventries}
